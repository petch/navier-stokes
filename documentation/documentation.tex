\documentclass[a4paper,10pt]{report}

\usepackage[utf8]{inputenc}
\usepackage{amsmath}
\usepackage{amssymb}
\usepackage{amsfonts}
\usepackage{bm}
\usepackage{tikz}
\usepackage{float}
\usepackage{pgfplots}
\usepackage{algorithm}
\usepackage{listings}
\usepackage[noend]{algpseudocode}
\usepackage[english,russian]{babel}


\title{Течение несжимаемой жидкости}
\author{Захаров ПЕ}

\begin{document}

\maketitle

\begin{abstract}
...
\end{abstract}



\chapter{Стационарное течение несжимаемой жидкости}

\section*{Введение}
Рассматривается численное моделирование стационарного течения несжимаемой жидкости. 

\section{Математические модели описания течений несжимаемой жидкости}

\subsection{Стационарное уравнение Навье-Стокса}

Течение жидкости описывается системой уравнений состоящей из уравнения для момента импульса и уравнения неразрывности. Для описания стационарного течения несжимаемой жидкости используются следующие уравнения в области $\Omega$
\begin{equation}
  \rho \,\mathcal{C}(\bm{u})
- \nabla \cdot \sigma(\bm{u}, p)
= \rho \bm{f},
\label{eq:navier-stokes}
\end{equation}
\begin{equation}
\nabla \cdot \bm{u} = 0,
\label{eq:incompressibility}
\end{equation}
где $\rho$ --- плотность, $\bm{u}(\bm{x})$ --- скорость жидкости, $\mathcal{C}(\bm{u})$ --- конвекция скорости, $\nabla$ --- оператор набла, $\cdot$ --- скалярное произведение, $\sigma(\bm{u}, p)=2 \mu \epsilon(\bm{u}) - p I$ --- тензор напряжения, $\mu$ --- вязкость, $\epsilon(\bm{u}) = \frac{1}{2}\left(\nabla \bm{u} + \nabla \bm{u}^T \right)$ --- тензор скорости деформации, $p$ --- давление, $I$ --- единичный оператор, $\bm{f}(\bm{x})$ --- объемные силы.

\subsection{Формы конвективной части}

Конвективную часть уравнения $\mathcal{C}(\bm{u})$ можно записывать в четырех различных формах, которые идентичны для непрерывных дифференциальных операторов, но различаются для дискретных операторов:
\begin{itemize}
\item $\nabla \cdot \left(\bm{u} \otimes \bm{u} \right)$ --- дивергентная форма;
\item $\bm{u} \cdot \nabla \bm{u}$ --- адвективная форма;
\item $\frac{1}{2}\left(\nabla \cdot \left(\bm{u} \otimes \bm{u} \right) + \bm{u} \cdot \nabla \bm{u} \right)$ --- кососимметричная форма;
\item $\left( \nabla \times \bm{u} \right) \times \bm{u} + \frac{1}{2} \nabla \left(\bm{u} \cdot \bm{u} \right) $ --- ротационная форма (rotation form);
\end{itemize}
где $\otimes$ --- тензорное произведение, $\times$ --- векторное произведение для трехмерной области, а для двумерной области $\nabla \times \bm{u} := -\frac{\partial u_1}{\partial x_2} + \frac{\partial u_2}{\partial x_1}$ и $a \times \bm{u} := (-a u_2, a u_1)^T$ для скаляра $a$


\subsection{Граничные условия}
Обозначим границы где задаются условия Дирихле для скорости через $\Gamma_d$, а с условиями Неймана -- $\Gamma_N$. Для задач течения жидкости стандартными являются следующие граничные условия
\begin{itemize}
\item условие прилипания (noslip condition) $\bm{u}(\bm{x}, t) = (0, 0)$;
\item условие скольжения или симметрии (slip condition) $\bm{u} \cdot \bm{n} = 0$;
\item условие на входе $\bm{u}(\bm{x}) = \bm{u}_{in}(\bm{x})$;
\item условие на выходе $p(\bm{x}) = p_{out}(\bm{x})$, $\nabla \bm{u} \cdot \bm{n} = 0$;
\item периодические граничные условия $\bm{u}(\bm{x}) = \bm{u}(\bm{x} + \bm{x}_{period})$;
\end{itemize}
где $\bm{n}$ --- внешняя нормаль границы. В некоторых задачах граничное условие для давления может не задаваться и тогда используется ограничение 
\begin{equation}
\int_\Omega p \; {\rm d}\bm{x} = p_{a},
\label{eq:constrain}
\end{equation}
которое дает единственное решение для давления.

\section{Вычислительный алгоритм}

\subsection{Вариационная задача}

Для задачи без ограничения для давления (\ref{eq:constrain})  используется стандартная вариационная задача. 
Для описания вариацонной формулировки уравнений умножаем на тестовые функции, итегрируем по области и понижаем порядок производных. Вариационная задача заключается в нахождении функций $\bm{u} \in V, p \in Q$, которые удовлетворяют
\begin{equation}
\begin{gathered}
F = \int_\Omega \rho \,\mathcal{C}(\bm{u}) \cdot \bm{v} \;{\rm d}\bm{x} 
 + \int_\Omega \sigma(\bm{u}, p) \cdot \epsilon(\bm{v}) \;{\rm d}\bm{x} \\
 - \int_{\Gamma_N} \mu \left(\nabla \bm{u}^T \cdot \bm{n} + \bm{g} + pI \cdot \bm{n} \right) \cdot \bm{v} \;{\rm d}\bm{s} \\
 - \int_\Omega \rho \bm{f} \cdot \bm{v} \;{\rm d}\bm{x} 
 + \nabla \cdot \bm{u} \; q \;{\rm d}\bm{x} = 0, \quad \forall \bm{v} \in \widehat{V}, \forall q \in \widehat{Q},
\end{gathered}
\label{eq:nonlinear-form}
\end{equation}
где $V, \widehat{V} = H^1(\Omega)$ --- пространства Соболева для векторных функций, которые удовлетворяют граничным условиям Дирихле для скорости, $Q, \widehat{Q} = L^2(\Omega)$ ---  квадратично интегрируемые пространства скалярных функций, которые удовлетворяют условиям Дирихле для давления.

\subsection{Ограничение для давления}

Ограничение для давления добавляется в вариационную формулировку используя дополнительную неизвестную как множитель Лагранжа. Вариационная задача: нужно найти функции $\bm{u} \in V, p \in Q$, которые удовлетворяют
\begin{equation}
\begin{gathered}
F = \int_\Omega \rho \,\mathcal{C}(\bm{u}) \cdot \bm{v} \;{\rm d}\bm{x} 
 + \int_\Omega \sigma(\bm{u}, p) \cdot \epsilon(\bm{v}) \;{\rm d}\bm{x} \\
 - \int_{\Gamma_N} \mu \left(\nabla \bm{u}^T \cdot \bm{n} + \bm{g} + pI \cdot \bm{n} \right) \cdot \bm{v} \;{\rm d}\bm{s} \\
 - \int_\Omega \rho \bm{f} \cdot \bm{v} \;{\rm d}\bm{x} 
 + \nabla \cdot \bm{u} \; q \;{\rm d}\bm{x} \\
 \int_\Omega \left(p - p_a\right) r \;{\rm d}\bm{x} + \int_\Omega c q \;{\rm d}\bm{x} = 0, \quad \forall \bm{v} \in \widehat{V}, \forall q \in \widehat{Q}, \forall r \in R,
\end{gathered}
\label{eq:nonlinear-form-constrained}
\end{equation}
где $R$ --- множество действительных чисел.

\subsection{Конечно-элементная аппроксимация}

При дискретизации используются труегольные сетки для двумерных задач и тетраэдральные для трехмерных задач.
Для аппроксимации скорости используется кусочно-квадратичные функции, а для давления --- кусочно-линейные функции, данная комбинация элементов известна как элементы Тэйлор-Худ.

\section{Программная реализация}
...

\subsection{Общее описание}
...

\subsection{Структура ПО}
...

\subsection{Ключевые элементы (описание классов)}
...

\subsection{Листинг программы}
...

\section{Тестирование ПО}
...

\subsection{Тестовые задачи (для демонстрации сходимости в зависимости от вычислительных параметров, ...)}
Рассмотрим стационарную задачу в единичном квадрате $\Omega=(0,1)^2$ с точным решением
\begin{equation}
\begin{gathered}
\bm{u}_e = \left(80 e^{x_1}(x_1 - 1)^2 x_1^2 x_2 (x_2 -  1) (2 x_2 - 1),\right.\\
\left. -40 e^{x_1} (x_1 - 1) x_1 (x_1 (3 + x_1) - 2) (x_2-1)^2 x_2^2\right), \\
p_e = 10 (-424 + 156 e + (x_2^2-x_2) (-456 + e^x_1) (456 + x_1^2) (228 - 5 (x_2^2 - x_2)) \\
+ 2 x_1 (-228 + (x_2^2 - x_2)) + 2 x_1^3 (-36 + x_2^2 - x_2)) + x_1^4 (12 + (x_2^2 - x_2))))).
\end{gathered}
\label{eq:shiftedvortex}
\end{equation}
Граничное и дополнительное условия
\[
    \bm{u} = \bm{u}_e, \quad \bm{x} \in \partial \Omega, \quad \int_\Omega p \,{\rm d}\bm{x} = 0.
\]
Возьмем следующие параметры модели
\[
\rho = 1, \quad \mu = 0.01, \quad \bm{f} = \rho \,\mathcal{C}(\bm{u}_e) - \nabla \cdot \epsilon(\bm{u}_e, p_e).
\]

\begin{figure}[H]
    \begin{minipage}{0.49\linewidth}
        \includegraphics[height=0.75\linewidth]{shiftedvortex/u.png}
    \end{minipage}
    \begin{minipage}{0.49\linewidth}
        \includegraphics[height=0.75\linewidth]{shiftedvortex/p.png}
    \end{minipage}
    \label{fig:shiftedvortex}
    \caption{Поле скорости и распределение давления}
\end{figure}
    
\begin{figure}[H]
    \begin{minipage}{0.49\linewidth}
        \includegraphics[height=0.75\linewidth]{shiftedvortex/eu.png}
    \end{minipage}
    \begin{minipage}{0.49\linewidth}
        \includegraphics[height=0.75\linewidth]{shiftedvortex/ep.png}
    \end{minipage}
    \label{fig:shiftedvortex-error}
    \caption{Погрешности скорости и давления}
\end{figure}

\subsection{Сходимость аппроксимации по пространству}

Для исследования численного решения от размера вычислительной сетки рассматриваем последовательность сеток: $8\times8$, $16\times16$, $32\times32$, $64\times64$, $128\times128$. Сравниваются различные нормы погрешности скорости и давления: 
\[
\left\Vert \bm{u} - \bm{u}_e \right\Vert_{L^2}, \quad
\left\Vert \bm{u} - \bm{u}_e \right\Vert_{L^\infty}, \quad
\left\Vert \bm{u} - \bm{u}_e \right\Vert_{H^{div}_0},
\]
\[
\left\Vert p - p_e \right\Vert_{L^2}, \quad
\left\Vert p - p_e \right\Vert_{L^\infty}, \quad
\left\Vert p - p_e \right\Vert_{H^1_0}.
\]

\begin{figure}[H]
\centering
\begin{tikzpicture}[scale=1]
\begin{axis}[
	xmode=log,
	ymode=log,
    ylabel={$\varepsilon$},
    xlabel=$N$,
    grid=major,
    legend pos=north east]
    \addplot[mark=*,solid] table[y=eul2]{shiftedvortex/meshes.txt};
    \addplot[mark=*,dashed] table[y=eumax]{shiftedvortex/meshes.txt};
    \addplot[mark=*,dotted] table[y=euhd0]{shiftedvortex/meshes.txt};
    \addlegendentry{$L^2$}
    \addlegendentry{$L^\infty$}
    \addlegendentry{$H^{div}_0$}
\end{axis}
\end{tikzpicture}
\caption{Зависимость нормы погрешности скорости от размера сетки $N \times N$}
\label{fig:shiftedvortex-meshes-u}
\end{figure}

\begin{figure}[H]
\centering
\begin{tikzpicture}[scale=1]
\begin{axis}[
	xmode=log,
	ymode=log,
    ylabel={$\varepsilon$},
    xlabel=$N$,
    grid=major,
    legend pos=north east]
    \addplot[mark=*,solid] table[y=epl2]{shiftedvortex/meshes.txt};
    \addplot[mark=*,dashed] table[y=epmax]{shiftedvortex/meshes.txt};
    \addplot[mark=*,dotted] table[y=eph10]{shiftedvortex/meshes.txt};
    \addlegendentry{$L^2$}
    \addlegendentry{$L^\infty$}
    \addlegendentry{$H^1_0$}
\end{axis}
\end{tikzpicture}
\caption{Зависимость нормы погрешности давления от размера сетки $N \times N$}
\label{fig:shiftedvortex-meshes-p}
\end{figure}

\subsection{Прямые линейные решатели}

Из сравнения прямых решателей (см. рис. \ref{fig:direct-solvers}) наилучшее время показал MUMPS, который имеет параллельное решение системы. На рис. \ref{fig:mumps-parallel} приводится сравнение времени параллельного решения решателем MUMPS для вычислительных сеток $128\times128$ и $256\times256$.

\begin{figure}[H]
\centering
\begin{tikzpicture}[scale=1]
\begin{axis}[
	xmode=log,
	ymode=log,
    ylabel={$t, s$},
    xlabel=$N$,
    grid=major,
    legend pos=north west]
    \addplot[mark=*,solid] table[y=mumps]{shiftedvortex/direct-solvers.txt};
    \addplot[mark=*,dashed] table[y=petsc]{shiftedvortex/direct-solvers.txt};
    \addplot[mark=*,dashdotted] table[y=umfpack]{shiftedvortex/direct-solvers.txt};
    \addplot[mark=*,dotted] table[y=superlu]{shiftedvortex/direct-solvers.txt};
    \addlegendentry{MUMPS}
    \addlegendentry{PETSc}
    \addlegendentry{UMFPack}
    \addlegendentry{SuperLU}
\end{axis}
\end{tikzpicture}
\caption{Время решения прямых решателей от размера сетки $N \times N$}
\label{fig:direct-solvers}
\end{figure}

\subsection{Итерационные линейные решатели}
...

\subsection{Преобуславливатели линейных решателей}
...

\subsection{Результаты по параллелизации}

\begin{figure}[H]
    \centering
    \begin{tikzpicture}[scale=1]
    \begin{axis}[
        xmode=log,
        ymode=log,
        ylabel={$t, s$},
        xlabel=$P$,
        grid=major,
        legend pos=north east]
        \addplot[mark=*,solid] table[y=N128]{shiftedvortex/mumps-parallel.txt};
        \addplot[mark=*,dashed] table[y=N256]{shiftedvortex/mumps-parallel.txt};
        \addplot[mark=*,dotted] table[y=N512]{shiftedvortex/mumps-parallel.txt};
        \addlegendentry{$N=128$}
        \addlegendentry{$N=256$}
        \addlegendentry{$N=512$}
    \end{axis}
    \end{tikzpicture}
    \caption{Время решения решателя MUMPS от количества процессов $P$}
    \label{fig:mumps-parallel}
\end{figure}


\pagebreak
\section{Верификация ПО}
...

\subsection{Матрица верификации (общее описание набор задач)}
...

\subsection{Задача 1}
Рассмотрим течение Пуазейля в единичном квадрате. 
Будем использовать следующее точное решение, которое описывает лиминарное течение между двумя пластинами
\[
    \bm{u}_e = \left(0.5/\mu x_1 (1-x_1), 0 \right), \quad
    p_e = 1-x_0,
\]
Сверху и снизу области находятся пластины (твердые стенки), слева и справа задаем давления. Граничные условия будут следующими
\[
    \bm{u} = (0, 0), \quad x_1 = 0, 1, \quad 
    p = 1, \quad x_0 = 0, \quad 
    p = 0, \quad x_0 = 1.
\]
Параметры модели 
\[
    \rho=1, \quad \mu = 1, \quad \bm{f}= \rho \,\mathcal{C}(\bm{u}_e) - \nabla \cdot \epsilon(\bm{u}_e, p_e).
\]

\begin{figure}[H]
    \begin{minipage}{0.49\linewidth}
        \includegraphics[height=0.75\linewidth]{poiseuilleflow/u.png}
    \end{minipage}
    \begin{minipage}{0.49\linewidth}
        \includegraphics[height=0.75\linewidth]{poiseuilleflow/p.png}
    \end{minipage}
    \label{fig:poiseuilleflow}
    \caption{Поле скорости и распределение давления}
\end{figure}
    
\begin{figure}[H]
    \begin{minipage}{0.49\linewidth}
        \includegraphics[height=0.75\linewidth]{poiseuilleflow/eu.png}
    \end{minipage}
    \begin{minipage}{0.49\linewidth}
        \includegraphics[height=0.75\linewidth]{poiseuilleflow/ep.png}
    \end{minipage}
    \label{fig:poiseuilleflow-error}
    \caption{Погрешности скорости и давления}
\end{figure}


\subsection{Задача 2}

Рассмотрим другую задачу в единичном квадрате, которая называется течением Лукаса. Точное решение 
\[
    \bm{u}_e = (-\cos(\pi x_0)/\pi, -x_1 \sin(\pi x_0)), \quad
    p_e = 0.
\]
Граничные и дополнительные условия
\[
    \bm{u} = \bm{u}_e, \quad \bm{x} \in \partial \Omega, \quad \int_\Omega p \,{\rm d}\bm{x} = 0.
\]
Параметры модели 
\[
    \rho=1, \quad \mu = 1, \quad \bm{f}=\rho \,\mathcal{C}(\bm{u}_e) - \nabla \cdot \epsilon(\bm{u}_e, p_e).
\]

\begin{figure}[H]
    \begin{minipage}{0.49\linewidth}
        \includegraphics[height=0.75\linewidth]{lucasflow/u.png}
    \end{minipage}
    \begin{minipage}{0.49\linewidth}
        \includegraphics[height=0.75\linewidth]{lucasflow/p.png}
    \end{minipage}
    \label{fig:lucasflow}
    \caption{Поле скорости и распределение давления}
\end{figure}
    
\begin{figure}[H]
    \begin{minipage}{0.49\linewidth}
        \includegraphics[height=0.75\linewidth]{lucasflow/eu.png}
    \end{minipage}
    \begin{minipage}{0.49\linewidth}
        \includegraphics[height=0.75\linewidth]{lucasflow/ep.png}
    \end{minipage}
    \label{fig:lucasflow-error}
    \caption{Погрешности скорости и давления}
\end{figure}

\subsection{Задача 3}

Рассмотрим задачу в единичном квадрате с центром в начале координат $\Omega = (-0.5, 0.5)^2$. Данное точное решение называется вихрем Тейлора 
\[
    \bm{u}_e = \left(-\cos(\pi x_0) \sin(\pi x_1), \sin(\pi x_0) \cos(\pi x_1)\right),
\]
\[
    p_e = -\rho/4 (\cos(2 \pi x_0) + \cos(2 \pi x_1)).
\]
Граничные и дополнительные условия
\[
    \bm{u} = \bm{u}_e, \quad \bm{x} \in \partial \Omega, \quad \int_\Omega p \,{\rm d}\bm{x} = 0.
\]
Параметры модели 
\[
    \rho=1, \quad \mu = 1, \quad \bm{f}=\rho \,\mathcal{C}(\bm{u}_e) - \nabla \cdot \epsilon(\bm{u}_e, p_e).
\]

\begin{figure}[H]
    \begin{minipage}{0.49\linewidth}
        \includegraphics[height=0.75\linewidth]{taylorvortex/u.png}
    \end{minipage}
    \begin{minipage}{0.49\linewidth}
        \includegraphics[height=0.75\linewidth]{taylorvortex/p.png}
    \end{minipage}
    \label{fig:taylorvortex}
    \caption{Поле скорости и распределение давления}
\end{figure}
    
\begin{figure}[H]
    \begin{minipage}{0.49\linewidth}
        \includegraphics[height=0.75\linewidth]{taylorvortex/eu.png}
    \end{minipage}
    \begin{minipage}{0.49\linewidth}
        \includegraphics[height=0.75\linewidth]{taylorvortex/ep.png}
    \end{minipage}
    \label{fig:taylorvortex-error}
    \caption{Погрешности скорости и давления}
\end{figure}


\subsection{Задача 4}

Рассмотрим задачу в единичном квадрате без точного решения. Снизу и слева расположены твердые стенки
\[
    \bm{u} = (0, 0), \quad x_0 = 0, \quad \bm{u} = (0, 0), \quad x_1 = 0,
\]
сверху задаем профиль скорости
\[
    \bm{u} = \left(0, -\sin(\pi (x_0^3 - 3 x_0^2 + 3 x_0))\right), \quad x_1 = 1,
\]
а справа условие на выходе
\[
    \nabla \bm{u} \cdot \bm{n} = 0, \quad x_0 = 1.
\]
Также добавляем дополнительное условие для давления
\[
    \int_\Omega p \,{\rm d}\bm{x} = 0.
\]
Параметры модели 
\[
    \rho=1, \quad \mu = 1, \quad \bm{f}=(0, 0).
\]

\begin{figure}[H]
    \begin{minipage}{0.49\linewidth}
        \includegraphics[height=0.75\linewidth]{kanflow/u.png}
    \end{minipage}
    \begin{minipage}{0.49\linewidth}
        \includegraphics[height=0.75\linewidth]{kanflow/p.png}
    \end{minipage}
    \label{fig:kanflow}
    \caption{Поле скорости и распределение давления}
\end{figure}
    

\subsection{Задача 5}

Рассмотрим задачу течения в канале с длиной $L=1$ и высотой широкой части $H=0.4$. Слева расположена узкая часть канала с высотой $h=0.2$, где задаем граничное условие на входе
\[
    \bm{u} = \left( (h - x_1) (h + x_1)/h^2, 0 \right), \quad x_0 = 0.
\]
На нижней границе задаем условие симметрии или по другому скольжения
\[
    \bm{u}_1 = 0, \quad x_1 = 0.
\]
Справа определяем условие на выходе
\[
    \nabla \bm(u) \cdot \bm{n} = (0, 0), \quad x_0 = L, \quad
    p = 0, \quad x_0 = L.
\]
Оставщуюйся верхнюю часть границы обозначим через $\Gamma_t$ и задаем условие твердой стенки
\[
    \bm{u} = (0, 0), \quad \bm{x} \in \Gamma_t.
\]
Параметры модели 
\[
    \rho=1, \quad \mu = 1, \quad \bm{f}=(0, 0).
\]

\begin{figure}[H]
    \begin{minipage}{0.49\linewidth}
        \includegraphics[height=0.75\linewidth]{channelflow/u.png}
    \end{minipage}
    \begin{minipage}{0.49\linewidth}
        \includegraphics[height=0.75\linewidth]{channelflow/p.png}
    \end{minipage}
    \label{fig:channelflow}
    \caption{Поле скорости и распределение давления}
\end{figure}
 

%\bibliographystyle{unsrt}
%\bibliography{literature}

\end{document}
